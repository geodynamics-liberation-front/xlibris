% \iffalse
%% File xoutput.dtx (C) Copyright 1999-2000 Frank Mittelbach, David Carlisle, Chris Rowley
%%                  (C) Copyright 2001,2004-2007 Frank Mittelbach, The LaTeX3 Project
%%                  (C) Copyright 2008-2012 The LaTeX3 Project
%%
%% It may be distributed and/or modified under the conditions of the
%% LaTeX Project Public License (LPPL), either version 1.3c of this
%% license or (at your option) any later version.  The latest version
%% of this license is in the file
%%
%%    http://www.latex-project.org/lppl.txt
%%
%% This file is part of the ``xor bundle'' (The Work in LPPL)
%% and all files in that bundle must be distributed together.
%%
%% The released version of this bundle is available from CTAN.
%%
%% -----------------------------------------------------------------------
%%
%% The development version of the bundle can be found at
%%
%%    http://www.latex-project.org/svnroot/experimental/trunk/
%%
%% for those people who are interested.
%%
%%%%%%%%%%%
%% NOTE: %%
%%%%%%%%%%%
%%
%%   Snapshots taken from the repository represent work in progress and may
%%   not work or may contain conflicting material!  We therefore ask
%%   people _not_ to put them into distributions, archives, etc. without
%%   prior consultation with the LaTeX Project Team.
%%
%% -----------------------------------------------------------------------
%%
\RequirePackage{l3bootstrap}
\GetIdInfo $Id: xoutput.dtx 3991 2012-07-16 19:00:35Z joseph $
          {xoutput}
\ProvidesExplPackage{\ExplFileName}
  {\ExplFileDate}{\ExplFileVersion}{\ExplFileDescription}
% \fi
%
%
%    \begin{macrocode}
\RequirePackage{
  calc,
  expl3
}
% this needs a bit of further thought and the names may have to change. the
% basic idea that if you want to base the default value of some tl kind of
% key on some other key value then right now this can't be directly specified in
% the template (perhaps it should). so what this below does is the following:
% we set the key to a value which one can't specify with  |\tl_set:Nn| (i.e.,
% basically invalid) and later test if this value is still the case.

\cs_new_nopar:Npn \TP_unset_key:N #1 {
    \cs_set_eq:NN #1 \scan_stop:
}

\cs_new_eq:NN \if_TP_key_unset:NT \cs_if_free:NT
\cs_new_eq:NN \if_TP_key_unset:NF \cs_if_free:NF

\RequirePackage {
  xtemplate,
  xmarks,
  xparse,
  xo-trace,xo-or,
  xo-here,
  xo-place,xo-page,xo-footnote,xo-float,xo-capt,xo-final,
  xo-new,
}
%    \end{macrocode}
%
%
%
% \endinput
\endinput
