\documentclass[a4paper, 10pt]{article}
\usepackage[utf8]{inputenc}
\usepackage[T1]{fontenc}
\usepackage{textcomp}
\usepackage{lmodern}
\usepackage{fixltx2e}

\usepackage{multicol, geometry, ragged2e, shortvrb, fancyhdr, titlesec, xspace}
\usepackage[babel=true, expansion=false]{microtype}
\usepackage[british, frenchb]{babel}
\usepackage[colorlinks=true]{hyperref}

\frenchbsetup{AutoSpacePunctuation=false, og=«, fg=»}
\newcommand\eng[1]{\foreignlanguage{english}{\emph{#1}}}

\geometry{hmargin=1.2cm, vmargin=2.4cm, includefoot}
\pagestyle{empty}

\setcounter{secnumdepth}{-1}
\titleformat\section{\sffamily\slshape\large}{}{0pt}{}
\titleformat\subsection[runin]{\bfseries}{}{0pt}{}[.]
\titlespacing\section{0pt}{\medskipamount}{\medskipamount}
\titlespacing\subsection{0pt}{\medskipamount}{1em}

\fancypagestyle{firstpage}{%
  \fancyhf{}
  \renewcommand\headrulewidth{0pt}
  \renewcommand\footrulewidth{0pt}
  \cfoot{\parbox\linewidth{\footnotesize
  Les nouvelles \latexx et le logiciel \latex vous sont fournis par l'équipe
  du projet LaTeX3 ; copyright 2009, tous droits réservés.\\
  Traduction française par Manuel \bsc{Pégourié-Gonnard} avec l'aimable
  autorisation de l'équipe du projet \latexx.\\
  La traduction se veut fidèle, mais seul l'original, disponible sur
  \url{http://www.latex-project.org/l3news/}, fait autorité.}}
  }

% #1 = Nième édition, mois AAAA
\newcommand\lnewsbegin[1]{%
  \RaggedRight
  \Huge
  Nouvelles de \latexx
  \\ \normalsize
  #1%
  \par \vspace{1pc}
  \thispagestyle{firstpage}
  \begin{multicols*}{2} \setlength\parindent{1pc}
  }

\newcommand\lnewsend{%
  \end{multicols*}
  }

\newcommand\newcolumn{\vfill\columnbreak}
\newcommand\tsvp{\par\medskip
  {\raggedleft \footnotesize (T. S. V. P.) \par}
  \newcolumn}

\newcommand\tex{\TeX\xspace}
\newcommand\latex{\LaTeX\xspace}
\newcommand\latexe{\LaTeXe\xspace}
\newcommand\latexx{\LaTeX3\xspace}
\newcommand\xetex{Xe\TeX\xspace}
\newcommand\luatex{Lua\TeX\xspace}
\newcommand\texlive{\TeX\,Live\xspace}
\newcommand\miktex{MiK\TeX\xspace}

\newcommand\fnurl[1]{\footnote{\url{#1}}}
\newcommand\pk{\textsf}
\MakeShortVerb\|

\endinput


\begin{document}

\lnewsbegin{Deuxième édition, juin 2009}

\section{\TeX\ Live et le code d'expl3}

\TeX\ Live 2009 arrive rapidement, et l'équipe \latexx a préparé une nouvelle
version du code expérimental de \latexx pour cette échéance. Des changements
très importants ont été faits depuis la dernière version publique du code,
dans \TeX\ Live 2008 ; aucune compatibilité ascendante n'a été maintenue (comme
en avertissait la documentation) mais nous pensons que les modifications
apportées sont toutes bénéfiques. Chaque partie ou presque de \pk{expl3} a été
scrutée, avec pour résultat une base de code bien plus cohérente.

Le code d'\pk{expl3} est maintenant considéré comme bien plus stable qu'il ne
l'était avant ; une suite de tests exhaustive a été écrite, qui nous aidera à
être sûrs que nous ne faisons pas d'erreur en modifiant les choses à l'avenir.
Pendant l'écriture de cette suite de tests, plusieurs bogues mineurs ont été
corrigés ; nous recommandons ce genre de suite pour tous les projets de
développement similaires ! Quelques petits changements discrets sont encore
attendus dans le code d'\pk{expl3}, mais aucun changement majeur ou rupture ne
sont prévus.

\section{Mise à jour planifiées}

Jusqu'à présent, la dernière mise à jour sur le CTAN de la suite \pk{expl3}
était pour \TeX\ Live 2008. Maintenant que le travail sur le code se fait de
façon plutôt stable, nous prévoyons de sortir des versions sur le CTAN plus
fréquemment. Ceci permettra à tout personne souhaitant expérimenter avec le
nouveau code d'utiliser les gestionnaires de paquets de \texlive ou \miktex
pour installer une version récente, sans avoir à récupérer une copie de
travail du dépôt SVN puis installer les modules à la main.

\section{Nouveaux membres}

Nous n'avons rien dit à ce sujet dans le dernier bulletin de nouvelles, mais
Joseph \bsc{Wright} et Will \bsc{Roberstson} sont maintenant membres de
l'équipe \latex. Ils ont travaillé assez exclusivement sur le code
d'\pk{expl3}.

Il est utile de répéter que \latexe est essentiellement gelé de façon à éviter
des problèmes de compatibilité arrière. Bien qu'il soit souhaitable de
bénéficier des nouvelles fonctionnalités offertes par les nouveaux moteurs
\xetex et \luatex, nous ne pouvons pas risquer de compromettre la stabilité de
serveurs en production utilisant de vieilles versions de \latexe, qui vont
inévitablement finir par traiter des documents écrits dans le futur.

\latexx n'héritera pas des mêmes contraintes, donc restez à l'écoute.

\newcolumn

\section{Quelques points spécifiques}

Morten \bsc{Høgholm} présentera les changements récents avec bien plus de
détails au TUG 2009. Voici quelques précisions rapides, pour ceux que cela
intéresse, sur le code nouvellement écrit et les changements d'envergure dans
les modules \pk{expl3}.

\subsection{Noms de fonctions plus logiques}

Plusieurs noms de fonctions qui restaient dans le système de noms de \TeX\ ont
été changés pour s'insérer dans le schéma plus logique de \pk{expl3} ; par
exemple, |\def:Npn| et |\let:NN| sont désormais |\cs_set:Npn| et
|\cs_set_eq:NN|.

\subsection{Définitions de fonctions et conditions}

Beaucoup de réflexion a été accordée aux nouvelles façons de définir des
fonctions et des conditions avec un minimum de code. Voir par exemple
|\cs_set:Nn| et |\prg_set_conditional:Nnn|.

\subsection{Comparaisons intelligentes}

Il est bien plus facile de faire des comparaisons maintenant, en utilisant
une syntaxe familière comme |\intexpr_compare_p:n{ #1+3 != \l_tmpa_int }|.

\subsection{Données depuis les variables}

Une nouvelle spécification d'argument de fonction |V| a été ajouté pour
extraire l'information de variables de différents types, sans avoir besoin de
connaître la structure sous-jacente de la variable. Un petit nettoyage des
spécifications d'arguments proposées a été fait, en partie comme conséquence
de l'ajout de celle-ci.

\subsection{l3msg}

Il y a un nouveau module pour gérer la communication entre le code \latexx et
l'utilisateur (messages d'information, avertissements et erreurs), avec un
filtrage des messages en partie inspiré du module \pk{silence}.

\section{Les six prochains mois}

Après avoir révisé le code d'\pk{expl3}, nous prévoyons maintenant d'appliquer
un processus similaire aux fondations des \pk{xpackages}. Ces derniers sont
une collection de modules de plus haut niveau qui répondent à des besoins de
base comme le contrôle de la disposition de la page et des interactions riches
avec l'utilisateur au niveau du document. Lorsque que le travail de fond pour
cette couche de traitement du document aura gagné en maturité, nous serons en
mesure de commencer à construire plus de modules pour un noyau \latexx ; ces
modules seront aussi utilisables sur \latexe et serviront de modèles largement
personnalisables pour la future conception de documents.

\tsvp

Lorsque des trous dans les fonctionnalités offertes par \pk{expl3} seront
découverts (dans certains cas, nous savons déjà qu'ils existent), cette couche
de programmation sera étendue pour satisfaire à nos besoins. Dans d'autres
cas, des enveloppes pour des fonctions \tex qui peuvent être utiles à un niveau
supérieur seront écrites.

\medskip

Concernant les points sur lesquels nous prévoyons de travailler ensuite, trois
modules de \pk{xpackage} vont retenir notre attention.

\subsection{xbase}

Actuellement, \pk{xbase} consiste en deux modules : \pk{xparse} et
\pk{template}, qui contiennent le code pour, respectivement, définir de
nouvelles commandes de niveau document (susceptibles d'être appelées par un
utilisateur ; p. ex. |\section|, |\makebox|, \dots) et pour gérer des listes
clé = valeur pour les entrées de l'utilisateur et la spécification du
document. \pk{xparse} a été présenté au TUG 1999\fnurl
{http://www.latex-project.org/papers/tug99.pdf} et Lars \bsc{Helström} a écrit
des notes sur \pk{template} en 2000\fnurl
{http://www.latex-project.org/papers/template-notes.pdf}. Ces modules offrent
un bon ensemble de fonctionnalités, mais les concepts ont besoin d'être
largement discutés. Plusieurs approches ont été adoptées pour la syntaxe clé =
valeur, dont certaines plus récentes que le code de \pk{template}, il y a donc
des alternatives à évaluer.

\subsection{galley2}

Gestion sophistiquée de la constructions des paragraphes et des autres
éléments du document. Morten en a parlé au TUG 2008\fnurl
{http://river-valley.tv/the-galley-module/}. La conception a besoin d'être
revue après quelques tests d'effort.

\subsection{xor}

Il s'agit de la routine de sortie \latexx, pour découper le \eng{galley} en
morceaux de la taille d'une page ou de sous-pages. Les idées et le code sont à
travailler pour atteindre le statut « utilisable en production ». Les premiers
développements de ce module ont été publiés par Frank en 2000\fnurl
{http://www.latex-project.org/papers/xo-pfloat.pdf}.

\medskip

Vous devriez nous entendre à nouveau vers Noël. Si vous avez envie de discuter
des idées présentées ici, vous pouvez nous rejoindre sur la liste de diffusion
\mbox{\textsc{latex-l}}\fnurl {http://www.latex-project.org/code.html}.

\lnewsend

\end{document}
